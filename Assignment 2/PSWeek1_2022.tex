% Options for packages loaded elsewhere
\PassOptionsToPackage{unicode}{hyperref}
\PassOptionsToPackage{hyphens}{url}
%
\documentclass[
]{article}
\usepackage{amsmath,amssymb}
\usepackage{lmodern}
\usepackage{iftex}
\ifPDFTeX
  \usepackage[T1]{fontenc}
  \usepackage[utf8]{inputenc}
  \usepackage{textcomp} % provide euro and other symbols
\else % if luatex or xetex
  \usepackage{unicode-math}
  \defaultfontfeatures{Scale=MatchLowercase}
  \defaultfontfeatures[\rmfamily]{Ligatures=TeX,Scale=1}
\fi
% Use upquote if available, for straight quotes in verbatim environments
\IfFileExists{upquote.sty}{\usepackage{upquote}}{}
\IfFileExists{microtype.sty}{% use microtype if available
  \usepackage[]{microtype}
  \UseMicrotypeSet[protrusion]{basicmath} % disable protrusion for tt fonts
}{}
\makeatletter
\@ifundefined{KOMAClassName}{% if non-KOMA class
  \IfFileExists{parskip.sty}{%
    \usepackage{parskip}
  }{% else
    \setlength{\parindent}{0pt}
    \setlength{\parskip}{6pt plus 2pt minus 1pt}}
}{% if KOMA class
  \KOMAoptions{parskip=half}}
\makeatother
\usepackage{xcolor}
\usepackage[margin=1in]{geometry}
\usepackage{color}
\usepackage{fancyvrb}
\newcommand{\VerbBar}{|}
\newcommand{\VERB}{\Verb[commandchars=\\\{\}]}
\DefineVerbatimEnvironment{Highlighting}{Verbatim}{commandchars=\\\{\}}
% Add ',fontsize=\small' for more characters per line
\usepackage{framed}
\definecolor{shadecolor}{RGB}{248,248,248}
\newenvironment{Shaded}{\begin{snugshade}}{\end{snugshade}}
\newcommand{\AlertTok}[1]{\textcolor[rgb]{0.94,0.16,0.16}{#1}}
\newcommand{\AnnotationTok}[1]{\textcolor[rgb]{0.56,0.35,0.01}{\textbf{\textit{#1}}}}
\newcommand{\AttributeTok}[1]{\textcolor[rgb]{0.77,0.63,0.00}{#1}}
\newcommand{\BaseNTok}[1]{\textcolor[rgb]{0.00,0.00,0.81}{#1}}
\newcommand{\BuiltInTok}[1]{#1}
\newcommand{\CharTok}[1]{\textcolor[rgb]{0.31,0.60,0.02}{#1}}
\newcommand{\CommentTok}[1]{\textcolor[rgb]{0.56,0.35,0.01}{\textit{#1}}}
\newcommand{\CommentVarTok}[1]{\textcolor[rgb]{0.56,0.35,0.01}{\textbf{\textit{#1}}}}
\newcommand{\ConstantTok}[1]{\textcolor[rgb]{0.00,0.00,0.00}{#1}}
\newcommand{\ControlFlowTok}[1]{\textcolor[rgb]{0.13,0.29,0.53}{\textbf{#1}}}
\newcommand{\DataTypeTok}[1]{\textcolor[rgb]{0.13,0.29,0.53}{#1}}
\newcommand{\DecValTok}[1]{\textcolor[rgb]{0.00,0.00,0.81}{#1}}
\newcommand{\DocumentationTok}[1]{\textcolor[rgb]{0.56,0.35,0.01}{\textbf{\textit{#1}}}}
\newcommand{\ErrorTok}[1]{\textcolor[rgb]{0.64,0.00,0.00}{\textbf{#1}}}
\newcommand{\ExtensionTok}[1]{#1}
\newcommand{\FloatTok}[1]{\textcolor[rgb]{0.00,0.00,0.81}{#1}}
\newcommand{\FunctionTok}[1]{\textcolor[rgb]{0.00,0.00,0.00}{#1}}
\newcommand{\ImportTok}[1]{#1}
\newcommand{\InformationTok}[1]{\textcolor[rgb]{0.56,0.35,0.01}{\textbf{\textit{#1}}}}
\newcommand{\KeywordTok}[1]{\textcolor[rgb]{0.13,0.29,0.53}{\textbf{#1}}}
\newcommand{\NormalTok}[1]{#1}
\newcommand{\OperatorTok}[1]{\textcolor[rgb]{0.81,0.36,0.00}{\textbf{#1}}}
\newcommand{\OtherTok}[1]{\textcolor[rgb]{0.56,0.35,0.01}{#1}}
\newcommand{\PreprocessorTok}[1]{\textcolor[rgb]{0.56,0.35,0.01}{\textit{#1}}}
\newcommand{\RegionMarkerTok}[1]{#1}
\newcommand{\SpecialCharTok}[1]{\textcolor[rgb]{0.00,0.00,0.00}{#1}}
\newcommand{\SpecialStringTok}[1]{\textcolor[rgb]{0.31,0.60,0.02}{#1}}
\newcommand{\StringTok}[1]{\textcolor[rgb]{0.31,0.60,0.02}{#1}}
\newcommand{\VariableTok}[1]{\textcolor[rgb]{0.00,0.00,0.00}{#1}}
\newcommand{\VerbatimStringTok}[1]{\textcolor[rgb]{0.31,0.60,0.02}{#1}}
\newcommand{\WarningTok}[1]{\textcolor[rgb]{0.56,0.35,0.01}{\textbf{\textit{#1}}}}
\usepackage{longtable,booktabs,array}
\usepackage{calc} % for calculating minipage widths
% Correct order of tables after \paragraph or \subparagraph
\usepackage{etoolbox}
\makeatletter
\patchcmd\longtable{\par}{\if@noskipsec\mbox{}\fi\par}{}{}
\makeatother
% Allow footnotes in longtable head/foot
\IfFileExists{footnotehyper.sty}{\usepackage{footnotehyper}}{\usepackage{footnote}}
\makesavenoteenv{longtable}
\usepackage{graphicx}
\makeatletter
\def\maxwidth{\ifdim\Gin@nat@width>\linewidth\linewidth\else\Gin@nat@width\fi}
\def\maxheight{\ifdim\Gin@nat@height>\textheight\textheight\else\Gin@nat@height\fi}
\makeatother
% Scale images if necessary, so that they will not overflow the page
% margins by default, and it is still possible to overwrite the defaults
% using explicit options in \includegraphics[width, height, ...]{}
\setkeys{Gin}{width=\maxwidth,height=\maxheight,keepaspectratio}
% Set default figure placement to htbp
\makeatletter
\def\fps@figure{htbp}
\makeatother
\setlength{\emergencystretch}{3em} % prevent overfull lines
\providecommand{\tightlist}{%
  \setlength{\itemsep}{0pt}\setlength{\parskip}{0pt}}
\setcounter{secnumdepth}{-\maxdimen} % remove section numbering
\ifLuaTeX
  \usepackage{selnolig}  % disable illegal ligatures
\fi
\IfFileExists{bookmark.sty}{\usepackage{bookmark}}{\usepackage{hyperref}}
\IfFileExists{xurl.sty}{\usepackage{xurl}}{} % add URL line breaks if available
\urlstyle{same} % disable monospaced font for URLs
\hypersetup{
  pdftitle={Problem Set Week 1},
  pdfauthor={Bruno Nadalic Sotic},
  hidelinks,
  pdfcreator={LaTeX via pandoc}}

\title{Problem Set Week 1}
\author{Bruno Nadalic Sotic}
\date{13/01/2023}

\begin{document}
\maketitle

{
\setcounter{tocdepth}{2}
\tableofcontents
}
\hypertarget{directions-for-the-student}{%
\subsection{Directions for the
student}\label{directions-for-the-student}}

\begin{itemize}
\tightlist
\item
  Put all R code in code chunks and verbal answers outside code chunks.
\item
  If you get a piece of R code to work, set the code chunk option
  eval=FALSE to ensure the document can still be knitted.
\item
  Use tidyverse functions whenever possible.
\item
  Comment your code to communicate your intentions.
\item
  Ensure that the R Markdown document knits without problems into a PDF
  or Word document.
\item
  Submit the R Markdown document on Canvas (under Assignments) before
  the deadline.
\end{itemize}

\begin{Shaded}
\begin{Highlighting}[]
\CommentTok{\#Load all libraries in this code chunk.}
\FunctionTok{library}\NormalTok{(tidyverse)}
\end{Highlighting}
\end{Shaded}

\begin{verbatim}
## -- Attaching packages --------------------------------------- tidyverse 1.3.1 --
\end{verbatim}

\begin{verbatim}
## v ggplot2 3.3.5     v purrr   0.3.4
## v tibble  3.1.6     v dplyr   1.0.8
## v tidyr   1.2.0     v stringr 1.4.0
## v readr   2.1.2     v forcats 0.5.1
\end{verbatim}

\begin{verbatim}
## -- Conflicts ------------------------------------------ tidyverse_conflicts() --
## x dplyr::filter() masks stats::filter()
## x dplyr::lag()    masks stats::lag()
\end{verbatim}

\begin{Shaded}
\begin{Highlighting}[]
\FunctionTok{library}\NormalTok{(dplyr)}
\FunctionTok{library}\NormalTok{(ggplot2)}
\FunctionTok{library}\NormalTok{ (readr)}
\end{Highlighting}
\end{Shaded}

\hypertarget{data}{%
\subsection{Data}\label{data}}

\textbf{sleepers.csv}. The effect of two drugs (variable \emph{group})
on the increase in hours of sleep compared to a control condition
(variable \emph{extra}) in 67 patients (variable \emph{ID}). The
patient's age group (\emph{age}, 1 = 18-25, 2 = 26-35, 3 = 36-50, 4 =
50+) is provided as well.

\hypertarget{questions}{%
\subsection{Questions}\label{questions}}

\begin{enumerate}
\def\labelenumi{\arabic{enumi}.}
\tightlist
\item
  Import the data in R with \texttt{readr::} and count the missing
  values on each variable using R code.
\end{enumerate}

\begin{Shaded}
\begin{Highlighting}[]
\CommentTok{\#import data, assuming the WD is set. }
\CommentTok{\#Alternatively: readr::read\_csv("C:/Users/sotic/Desktop/DWR1/sleeprs.csv")}

\NormalTok{sleepers }\OtherTok{\textless{}{-}}\NormalTok{ readr}\SpecialCharTok{::}\FunctionTok{read\_csv}\NormalTok{(}\StringTok{"sleepers.csv"}\NormalTok{)}
\end{Highlighting}
\end{Shaded}

\begin{verbatim}
## Rows: 134 Columns: 4
## -- Column specification --------------------------------------------------------
## Delimiter: ","
## chr (1): group
## dbl (3): ID, age, extra
## 
## i Use `spec()` to retrieve the full column specification for this data.
## i Specify the column types or set `show_col_types = FALSE` to quiet this message.
\end{verbatim}

\begin{Shaded}
\begin{Highlighting}[]
\CommentTok{\#Count missing values for each column}
\NormalTok{sleepers }\SpecialCharTok{\%\textgreater{}\%} 
  \FunctionTok{summarize\_all}\NormalTok{(}\FunctionTok{funs}\NormalTok{(}\FunctionTok{sum}\NormalTok{(}\FunctionTok{is.na}\NormalTok{(.))))}
\end{Highlighting}
\end{Shaded}

\begin{verbatim}
## Warning: `funs()` was deprecated in dplyr 0.8.0.
## Please use a list of either functions or lambdas: 
## 
##   # Simple named list: 
##   list(mean = mean, median = median)
## 
##   # Auto named with `tibble::lst()`: 
##   tibble::lst(mean, median)
## 
##   # Using lambdas
##   list(~ mean(., trim = .2), ~ median(., na.rm = TRUE))
## This warning is displayed once every 8 hours.
## Call `lifecycle::last_lifecycle_warnings()` to see where this warning was generated.
\end{verbatim}

\begin{verbatim}
## # A tibble: 1 x 4
##      ID group   age extra
##   <int> <int> <int> <int>
## 1     0     0     3     0
\end{verbatim}

\hypertarget{conclusion}{%
\subsubsection{Conclusion:}\label{conclusion}}

Only \emph{`Age'} seems to have three missing values.

\begin{longtable}[]{@{}llll@{}}
\toprule()
Grading & Max points & Awarded & \\
\midrule()
\endhead
Ex. 1 & 1 & & \\
\bottomrule()
\end{longtable}

\begin{enumerate}
\def\labelenumi{\arabic{enumi}.}
\setcounter{enumi}{1}
\tightlist
\item
  Use \texttt{ggplot2::} to create one figure containing a frequency
  polygon of sleep increase for each drug. Use different colors for the
  two polygons. Which drug seems to increase sleep more?
\end{enumerate}

\begin{Shaded}
\begin{Highlighting}[]
\CommentTok{\#Plot}
\FunctionTok{ggplot}\NormalTok{(}\AttributeTok{data =}\NormalTok{ sleepers, }\FunctionTok{aes}\NormalTok{(}\AttributeTok{x=}\NormalTok{extra, }\AttributeTok{fill=}\NormalTok{group, }\AttributeTok{color=}\NormalTok{group)) }\SpecialCharTok{+} 
  \FunctionTok{geom\_freqpoly}\NormalTok{(}\AttributeTok{binwidth=}\DecValTok{2}\NormalTok{) }\SpecialCharTok{+}
  \FunctionTok{scale\_fill\_manual}\NormalTok{(}\AttributeTok{name=}\StringTok{"Drug"}\NormalTok{, }\AttributeTok{values=}\FunctionTok{c}\NormalTok{(}\StringTok{"Drug 1"} \OtherTok{=} \StringTok{"red"}\NormalTok{, }\StringTok{"Drug 2"} \OtherTok{=} \StringTok{"blue"}\NormalTok{), }\AttributeTok{labels=}\FunctionTok{c}\NormalTok{(}\StringTok{"Drug 1"}\NormalTok{, }\StringTok{"Drug 2"}\NormalTok{)) }\SpecialCharTok{+}
  \FunctionTok{ggtitle}\NormalTok{(}\StringTok{"Frequency Polygon of Sleep Increase by Drug"}\NormalTok{) }\SpecialCharTok{+}
  \FunctionTok{theme}\NormalTok{(}\AttributeTok{legend.position =} \StringTok{"top"}\NormalTok{)}
\end{Highlighting}
\end{Shaded}

\includegraphics{PSWeek1_2022_files/figure-latex/unnamed-chunk-2-1.pdf}

\hypertarget{answer}{%
\subsubsection{Answer:}\label{answer}}

Based on visual inspection, both drugs seem to follow an innitial
pattern. However, Drug 1 seems to increase sleep more at the peak. It is
important to note that this might not be a correct conclusion. The graph
shows the frequency of extra sleep for each drug group, but it does not
provide information on the average extra sleep or the overall effect of
the drug. To determine which drug is more effective at increasing sleep,
we would need additional information such as the average increase in
sleep for each drug group and a statistical analysis comparing the two
groups. It is also important to note that the graph is based on
observational data, which may not be enough to draw a conclusion about
causality, a controlled experiment would be needed to establish
causality.

\begin{longtable}[]{@{}llll@{}}
\toprule()
Grading & Max points & Awarded & \\
\midrule()
\endhead
Ex. 2 & 1 & & \\
\bottomrule()
\end{longtable}

\begin{enumerate}
\def\labelenumi{\arabic{enumi}.}
\setcounter{enumi}{2}
\tightlist
\item
  For each age group, create a figure as in Question 2 using facetting.
  Again, use \texttt{ggplot2::}. Which drug works better (gives more
  additional sleep) for which age group?
\end{enumerate}

\begin{Shaded}
\begin{Highlighting}[]
\FunctionTok{ggplot}\NormalTok{(}\AttributeTok{data =}\NormalTok{ sleepers, }\FunctionTok{aes}\NormalTok{(}\AttributeTok{x =}\NormalTok{ extra, }\AttributeTok{fill =}\NormalTok{ group, }\AttributeTok{color =}\NormalTok{ group)) }\SpecialCharTok{+}
  \FunctionTok{geom\_freqpoly}\NormalTok{(}\AttributeTok{binwidth =} \DecValTok{2}\NormalTok{) }\SpecialCharTok{+}
  \FunctionTok{facet\_wrap}\NormalTok{(}\SpecialCharTok{\textasciitilde{}}\NormalTok{ age, }\AttributeTok{scales =} \StringTok{"free"}\NormalTok{) }\SpecialCharTok{+} \CommentTok{\#"free" parameter allows each facet to have its own scales.}
  \FunctionTok{ggtitle}\NormalTok{(}\StringTok{"Frequency Polygon of Sleep Increase by Drug and Age Group"}\NormalTok{) }\SpecialCharTok{+}
  \FunctionTok{scale\_fill\_manual}\NormalTok{(}\AttributeTok{name=}\StringTok{"Drug"}\NormalTok{, }\AttributeTok{values=}\FunctionTok{c}\NormalTok{(}\StringTok{"Drug 1"} \OtherTok{=} \StringTok{"red"}\NormalTok{, }\StringTok{"Drug 2"} \OtherTok{=} \StringTok{"blue"}\NormalTok{), }\AttributeTok{labels=}\FunctionTok{c}\NormalTok{(}\StringTok{"Drug 1"}\NormalTok{, }\StringTok{"Drug 2"}\NormalTok{))}
\end{Highlighting}
\end{Shaded}

\includegraphics{PSWeek1_2022_files/figure-latex/unnamed-chunk-3-1.pdf}

\hypertarget{answer-1}{%
\subsubsection{Answer}\label{answer-1}}

Based on the graphs, `Drug 1' seems to perform slightly better for age
groups in the categories of 1, 2 and 4. However, `Drug 2' works slightly
better for age group `3'. It is important to note that there are also
missing values under `Drug 2' as evident in the last plot. We could
exclude that data and plot from the analysis and visualization by adding
``na.rm = TRUE'' under ggplot(), but I decided to leave it as it
highlights whats lost.

\begin{longtable}[]{@{}llll@{}}
\toprule()
Grading & Max points & Awarded & \\
\midrule()
\endhead
Ex. 3 & 2 & & \\
\bottomrule()
\end{longtable}

\begin{enumerate}
\def\labelenumi{\arabic{enumi}.}
\setcounter{enumi}{3}
\tightlist
\item
  Use \texttt{dplyr::} to create a table (tibble/data frame) containing
  the number of observations, the minimum, maximum, and average sleep
  increase for each drug.
\end{enumerate}

\begin{Shaded}
\begin{Highlighting}[]
\NormalTok{sleepers }\SpecialCharTok{\%\textgreater{}\%}
  \FunctionTok{group\_by}\NormalTok{(group) }\SpecialCharTok{\%\textgreater{}\%}
  \FunctionTok{summarize}\NormalTok{(}\AttributeTok{n =} \FunctionTok{n}\NormalTok{(),}
            \AttributeTok{min =} \FunctionTok{min}\NormalTok{(extra, }\AttributeTok{na.rm =} \ConstantTok{TRUE}\NormalTok{),}
            \AttributeTok{max =} \FunctionTok{max}\NormalTok{(extra, }\AttributeTok{na.rm =} \ConstantTok{TRUE}\NormalTok{),}
            \AttributeTok{mean =} \FunctionTok{mean}\NormalTok{(extra, }\AttributeTok{na.rm =} \ConstantTok{TRUE}\NormalTok{))}
\end{Highlighting}
\end{Shaded}

\begin{verbatim}
## # A tibble: 2 x 5
##   group     n   min   max  mean
##   <chr> <int> <dbl> <dbl> <dbl>
## 1 Drug1    67 -2.07  5.98  1.61
## 2 Drug2    67 -2.72  4.98  1.32
\end{verbatim}

\begin{longtable}[]{@{}llll@{}}
\toprule()
Grading & Max points & Awarded & \\
\midrule()
\endhead
Ex. 1 & 1 & & \\
\bottomrule()
\end{longtable}

\begin{enumerate}
\def\labelenumi{\arabic{enumi}.}
\setcounter{enumi}{4}
\tightlist
\item
  Calculate the difference in extra sleep for each patient between drug
  2 and drug 1, again using \texttt{dplyr::}. Show the first 10 rows of
  the results.
\end{enumerate}

Tip: Sort the data and, in the end, only retain one case for each
patient.

\begin{Shaded}
\begin{Highlighting}[]
\NormalTok{sleepers\_diff }\OtherTok{\textless{}{-}}\NormalTok{ sleepers }\SpecialCharTok{\%\textgreater{}\%} 
  \FunctionTok{spread}\NormalTok{(group, extra) }\SpecialCharTok{\%\textgreater{}\%} 
  \FunctionTok{group\_by}\NormalTok{(ID) }\SpecialCharTok{\%\textgreater{}\%} 
  \FunctionTok{mutate}\NormalTok{(}\AttributeTok{extra\_diff =} \StringTok{\textasciigrave{}}\AttributeTok{Drug2}\StringTok{\textasciigrave{}} \SpecialCharTok{{-}} \StringTok{\textasciigrave{}}\AttributeTok{Drug1}\StringTok{\textasciigrave{}}\NormalTok{) }\SpecialCharTok{\%\textgreater{}\%} 
  \FunctionTok{select}\NormalTok{(ID, extra\_diff) }

\FunctionTok{head}\NormalTok{(sleepers\_diff,}\DecValTok{10}\NormalTok{)}
\end{Highlighting}
\end{Shaded}

\begin{verbatim}
## # A tibble: 10 x 2
## # Groups:   ID [10]
##       ID extra_diff
##    <dbl>      <dbl>
##  1     1     -4.59 
##  2     2      2.46 
##  3     3     -4.48 
##  4     4      2.27 
##  5     5     -2.12 
##  6     6     -0.590
##  7     7      3.94 
##  8     8     -0.385
##  9     9      3.09 
## 10    10     -0.385
\end{verbatim}

\begin{longtable}[]{@{}llll@{}}
\toprule()
Grading & Max points & Awarded & \\
\midrule()
\endhead
Ex. 5 & 2 & & \\
\bottomrule()
\end{longtable}

\begin{enumerate}
\def\labelenumi{\arabic{enumi}.}
\setcounter{enumi}{5}
\tightlist
\item
  Create a plot to explore the covariation (association) between two
  continuous variables in \emph{your project's data set} that are of
  interest to you. Use comments in the R code to justify the choices
  that you made to create this plot.
\end{enumerate}

Note: You may use grouped summaries (aggregation) to create continuous
variables.

\begin{Shaded}
\begin{Highlighting}[]
\CommentTok{\# install.packages("translate") \# install the package if not already installed}
\FunctionTok{library}\NormalTok{(translate)}

\CommentTok{\# Authorization through the API key}
\FunctionTok{Sys.setenv}\NormalTok{(}\StringTok{"google\_api\_credidentials"}\OtherTok{=}\StringTok{"path/to/your/key.json"}\NormalTok{) }\CommentTok{\#Authorize via account. Creditentials are personal and stored on my machine. Hence, this is left blank.}

\CommentTok{\# read in the German dataset}
\NormalTok{data }\OtherTok{\textless{}{-}} \FunctionTok{read.csv}\NormalTok{(}\StringTok{"NLResponders2009.csv"}\NormalTok{, }\AttributeTok{header =} \ConstantTok{TRUE}\NormalTok{, }\AttributeTok{sep =} \StringTok{","}\NormalTok{)}

\CommentTok{\# translate the data to English}
\NormalTok{translated\_data }\OtherTok{\textless{}{-}} \FunctionTok{translate}\NormalTok{(data, }\AttributeTok{source =} \StringTok{"de"}\NormalTok{, }\AttributeTok{target =} \StringTok{"en"}\NormalTok{)}

\CommentTok{\# write the translated data to a new CSV file}
\FunctionTok{write.csv}\NormalTok{(translated\_data, }\AttributeTok{file =} \StringTok{"Responders2009.csv"}\NormalTok{, }\AttributeTok{row.names =} \ConstantTok{FALSE}\NormalTok{)}
\end{Highlighting}
\end{Shaded}

\begin{longtable}[]{@{}llll@{}}
\toprule()
Grading & Max points & Awarded & \\
\midrule()
\endhead
Ex. 6 & 2 & & \\
Flawless knitting & 1 & & \\
\textbf{Total} & 10 & & \\
\bottomrule()
\end{longtable}

\end{document}
